
%%%%%%%%%%%%%%%%%%%%%%% file typeinst.tex %%%%%%%%%%%%%%%%%%%%%%%%%
%
% This is the LaTeX source for the TDPTemplate using
% the LaTeX document class 'llncs.cls' Springer LNAI format
% used in the RoboCup Symposium submissions.
% http://www.springer.com/computer/lncs?SGWID=0-164-6-793341-0
%
% It may be used as a template for your own TDP - copy it
% to a new file with a new name and use it as the basis
% for your Team Description Paper
%
% NB: the document class 'llncs' has its own and detailed documentation, see
% ftp://ftp.springer.de/data/pubftp/pub/tex/latex/llncs/latex2e/llncsdoc.pdf
%
%%%%%%%%%%%%%%%%%%%%%%%%%%%%%%%%%%%%%%%%%%%%%%%%%%%%%%%%%%%%%%%%%%%

\documentclass[runningheads,a4paper]{llncs}
\usepackage{amssymb}
\setcounter{tocdepth}{3}
\usepackage{graphicx}
\usepackage{amssymb}
\usepackage[utf8]{inputenc}
\usepackage{url}
\usepackage{float}
\usepackage{amsmath}
\usepackage{graphicx}
\usepackage{wrapfig}

\usepackage{lipsum}
\newcommand{\BnL}[1][1em]{ \includegraphics[width=#1]{images/bnl.jpg} }

\begin{document}

\title{Team-Name 2015 Team Description Paper}

\author{Team Leader \and Team Members }
\institute{[Intitute name and direction here], \\
\texttt{http://devoted-web-site.url}}
\maketitle


%%%%%%%%%%%%%%%%%%%%%%%%%%%%%%%%%%%%%%%%%%%%%%%%%%%%%%%%%%%%%%%%%%%%%%%%%%%%%%%%%%%%

\begin{abstract}

In your abstract, please state which is the main research line of your team for this year (on which problem or set of problems are you focusing all the team efforts). Tell why this research is important, how are you approaching to the problem solution and which results do you expect to obtain.

\end{abstract}


%%%%%%%%%%%%%%%%%%%%%%%%%%%%%%%%%%%%%%%%%%%%%%%%%%%%%%%%%%%%%%%%%%%%%%%%%%%%%%%%%%%%

\section{Introduction}
While writing the TDP, please focus on your current research and state clear its scientific contribution value and why it is important for you and the league. The length of the TDP is limited to 8 pages. Please leave the hardware and software description for the end of the paper.

Remember that the TDP must contain the following information:

\begin{itemize}
	\item Description of the hardware and software including a list of integrated externally available components (including commercial products, freeware, Open Source, etc.)
	\item Innovative technology and scientific contribution
	\item Photo(s) of the robot
	\item Focus of research/research interests
	\item Re-usability of the system for other research groups
	\item Applicability of the robot in the real world
\end{itemize}

\section{Background}
% We are Buy n Large. We have no competitors so no background is required.
\lipsum[1-3]

\section{BnL Trash Seeker Algorithm (Main research)}
\lipsum[4-14]

\section{Experiments and results}
\lipsum[15-20]

\section{Conclusions and future work}
\lipsum[21-24]

\section*{Robot WALL-E Hardware Description}
\textit{(In this section briefly describe the software and hardware of the robot)}

Robot WALL-E has the patented \textit{BnL Optimized Design} for garbage recollection. Specifications are as follows:

\begin{figure}
% \begin{wrapfigure}[13]{r}{0.4\textwidth}
	\centering
	\includegraphics[width=0.4\textwidth]{images/wall-e.jpg}
	\caption{Robot WALL-E}
	\label{fig:virbot}
% \end{wrapfigure}
\end{figure}


\begin{itemize}
	\item Base: BnL all-terrain base (differential pair), 2.5m/s max speed.
	\item Torso: BnL compressor with solar charger.
	\item Left and right arms: Mounted on torso. 7 DOF, anthropomorphic, BnL Design. Maximum load: 20kg.
	\item Neck: BnL telescopic neck with pan and tilt.
	\item Head: 1DOF BnL Expressive Eyes
	\item External devices: None
	\item Robot dimensions: height: 1.2m (max), width: 0.7m depth 0.8m
	\item Robot weight: 50kg.
\end{itemize}

\textit{Also our robot incorporates the following devices:}

\begin{itemize}
	\item \BnL Battery charge indicator
	\item \BnL Auto-focus all-purpose cameras
	\item \BnL 7DOF heavy duty fingers
	\item \BnL Cockroach
\end{itemize}

\section*{Robot's Software Description}
Please describe in this section the software you are using to control your robot.
Consider the following example:

\textit{For our robot we are using the following software:}

\begin{itemize}
	\item Platform: BnL Operating System
	\item Navigation, localization and mapping: BnL Navigator
	\item Face recognition: None. Not designed for human interaction.
	\item Speech recognition: BnL All-purpose recognizer. Please refer to [1, 2, 3]
	\item Speech generation: None. Not designed for human interaction.
	\item Object recognition: BnL Trash Seeker Algorithm. See description on previous sections.
	\item Arms control and two-hand coordination: BnL automatic controller. Please refer to [4, 5, 6]
\end{itemize}

\section*{Bibliography}
Add your references here
%\bibliographystyle{unsrt}
%\bibliography{bibliography}
\end{document} 