
%%%%%%%%%%%%%%%%%%%%%%% file typeinst.tex %%%%%%%%%%%%%%%%%%%%%%%%%
%
% Author: Mauricio Matamoros
% Updated: July 18, 2019
% Contact: mauricio@robocupathome.org
%
% This is the LaTeX source for the TDPTemplate using
% the LaTeX document class 'llncs.cls' Springer LNAI format
% used in the RoboCup Symposium submissions.
% http://www.springer.com/computer/lncs?SGWID=0-164-6-793341-0
%
% It may be used as a template for your own TDP - copy it
% to a new file with a new name and use it as the basis
% for your Team Description Paper
%
% NB: the document class 'llncs' has its own and detailed documentation, see
% ftp://ftp.springer.de/data/pubftp/pub/tex/latex/llncs/latex2e/llncsdoc.pdf
%
% Remark: Last page with specs won't be included in Camera ready TDP's.
%
% CHKTEX-FILE 8
% CHKTEX-FILE 13
%
%%%%%%%%%%%%%%%%%%%%%%%%%%%%%%%%%%%%%%%%%%%%%%%%%%%%%%%%%%%%%%%%%%%

\documentclass[runningheads,a4paper]{llncs}
\usepackage{amssymb}
\setcounter{tocdepth}{3}
\usepackage{graphicx}
\usepackage{amssymb}
\usepackage{enumitem}
\usepackage[utf8]{inputenc}
\usepackage[hidelinks]{hyperref}
\usepackage{url}
\usepackage{float}
\usepackage{amsmath}
\usepackage{graphicx}
\usepackage{wrapfig}
\usepackage{fancyhdr}
\usepackage{titling}
\usepackage{xcolor}
\usepackage{lipsum}
\newcommand{\robospecs}{%
	\newpage%
	\pagenumbering{gobble}%
	\pagestyle{fancy}%
	\fancyhf{}%
	\chead{$|$}
	\rhead{\footnotesize\thetitle}%
	\lhead{\footnotesize\theauthor}%
	\rfoot{Robot software and hardware specification sheet}%
}

\newcommand{\BnL}[1][1em]{ \includegraphics[width=#1]{images/bnl.jpg} }


%%%%%%%%%%%%%%%%%%%%%%%%%%%%%%%%%%%%%%%%%%%%%%%%%%%%%%%%%%%%%%%%%%%%%%%%%%%%%%%%%%%%
%
% Title
%
%%%%%%%%%%%%%%%%%%%%%%%%%%%%%%%%%%%%%%%%%%%%%%%%%%%%%%%%%%%%%%%%%%%%%%%%%%%%%%%%%%%%
\title{SkyNet 2020 Team Description Paper}

\author{Main-author \and Co-author \and Team Members}
\institute{Affiliation name and address, \\
\texttt{http://devoted-web-site.url}}


\begin{document}
\maketitle

%%%%%%%%%%%%%%%%%%%%%%%%%%%%%%%%%%%%%%%%%%%%%%%%%%%%%%%%%%%%%%%%%%%%%%%%%%%%%%%%%%%%
%
% Abstract
%
%%%%%%%%%%%%%%%%%%%%%%%%%%%%%%%%%%%%%%%%%%%%%%%%%%%%%%%%%%%%%%%%%%%%%%%%%%%%%%%%%%%%

\begin{abstract}
The present document serves as template for the 2020 Team Description Papers to be submitted for qualification in the 2020 RoboCup@Home international competition to be held in Bordeaux, France.

We strongly recommend abstracts to be limited to 250 words, and follow the three-paragraph rule: introduction and relevance, approach, and contributions. This will help your readers to get a better understanding of what you're about to present.

Your abstract must state your main research line and your scientific achievements of this year, namely which problems have been solved and where the research group focus is directed.
Summarize your achievements and why the conducted research is relevant in robotics.
Additionally, you may address your approach to the problem solution and which results do you expect to obtain.
\end{abstract}


%%%%%%%%%%%%%%%%%%%%%%%%%%%%%%%%%%%%%%%%%%%%%%%%%%%%%%%%%%%%%%%%%%%%%%%%%%%%%%%%%%%%

\section{Introduction}
The present document is a template for the Team Description Paper to be submitted for qualification in the 2020 RoboCup@Home international competition to be held in Bordeaux, France.

A Team Description Paper (hereinafter referred as TDP) is an 8-pages long scientific paper, detailing information on the technical and scientific approach of the team's research.
An updated version of this template can be download from {\small\url{https://github.com/RoboCupAtHome/TDPTemplate}}.
Contributions to enhancing this template (e.g.~typos, grammatical errors, etc.) are always welcome either as issues or pull requests.

\textbf{Copyright note:} All TDPs sent for qualification are made publicly available in the \href{https://github.com/RoboCupAtHome/AtHomeCommunityWiki/wiki}{RoboCup@Home Wiki} for further reference.
On submitting, teams implicitly grant permission to RoboCup@Home and the RoboCup Federation to copy, distribute, upload, publish, and use the manuscript to promote the event and the league at convenience.

\section{TDP contents}
While writing the TDP, focus on your current research, clearly stating all scientific contribution, and why are they important for you and the league. The length of the TDP is limited to 8 pages including references. \textbf{\color{red} Exceeding the number of pages will automatically void your application}.

The Team Description Paper shall use the \textit{Springer LNAI format}\footnotemark~used in the RoboCup Symposium submissions, and has a hard limit of 8 pages without altering margins or spacing (including references but excluding the annex).
Please notice that changes to the margins, space between paragraphs, and font size are not allowed (such TDP will be rejected). We suggest to leave the hardware and software description for the end of the paper in the annex.\footnotetext{\url{http://www.springer.com/computer/lncs?SGWID=0-164-6-793341-0}}

\textbf{Important Notice:} Attaching to the requested format is important for the camera ready version of the TDPs can be included in the memories of the competition.

\noindent The TDP must contain the following information:

\begin{itemize}[nosep]
	\item Innovative technology and scientific contribution
	\item Focus of research/research interests
	\item Re-usability of the system for other research groups
	\item Applicability of the robot in the real world
	\item \textbf{DSPL \& SSPL:} When the robot depicted in the TDP or Team Video is different from the league's standard one, the TDP must clearly state how the addressed approach and described software will be adapted to the standard platform robot.
\end{itemize}

The language for the whole TDP, including graphics, tables, images, and all additional content must be English. Content in other languages must be translated.


\section{Do's and Don'ts}
This section elaborates on a set of \textit{hits} about what to do (or what to avoid) when writing a TDP.
Following these guidelines may help your team to obtain qualification, although none is mandatory.
When writing a TDP, the best guidelines are the ones that help you to present your research in a clear, concise, and effective manner.

\subsection{No more than 8 pages}
Seriously. Exceeding the number of pages is the easiest way of forfeiting your qualification.

\subsection{Summarize your achievements}
In order to promote participation in local tournaments, experience is heavily weighted during the qualification process.
Dedicating one paragraph to summarize the experience and achievements of your team in local tournaments is therefore strongly advised.

\subsection{Save space for Science}
Describing your team in detail (e.g.~number of undergrads, grads, and PhD students) is of little interest for the league, and uses some valuable space that can be used instead to describe your research.
It is strongly advised that teams describe instead their approaches and achievements.
Our reviewers are far more interested in knowing how your robot manages to solve a task and reading about your ground-breaking discoveries, than in the history of the team.

\subsection{Avoid describing your robot}
Unless your research revolves around hardware design and implementation (e.g.~mechanical schematics, control, low-cost solutions), it is better to omit this section, specially in Standard Platform Leagues (the HW is already known to us).
Detailed software and hardware descriptions can be provided in the appendix, for which there is no page limit.

\subsection{Avoid explaining ROS}
With over 90\% usage in RoboCup@Home, it can be assumed that all reviewers are familiar with ROS.
Therefore, unless your research extends or enhances an specific ROS package, there is no need to explain how it works.
Moreover, TDPs revolving around the basics of ROS are typically rejected.

\subsection{Keep it simple}
Although reviewers share a common background in the domain of service robotics, is very unlikely that they are actively involved in your research field.
Therefore, you can't expect that we are familiar with the State-of-the-Art.
The organizing committee kindly asks authors to keep this in mind and write in a more descriptive and less analytic way.
The main goal of a TDP is tell others about your latest practical achievement, how your team managed to solve the problem, what strategy was chosen and why, while at the same time trying to convince your reader that what you are doing is useful or applicable in a daily life scenario.

\subsection{Mind your writing}
We strongly advise non-native speakers to double check the readability of their TDP's before submission.
More often than not, a team may not obtain qualification if the reviewers can't understand the ideas of the authors.

Likewise, and as a courtesy to our reviewers, we kindly ask native speakers to write as neutral as possible and refrain from using colloquial expressions and slang.
Although all reviewers are fluent English readers (even sometimes native speakers), they may not be proficient in a particular idiom such as Australian, British, or American English.
This situation may lead to ambiguities and misunderstandings which could reduce the submitter's score, reducing the chances of qualifying.

\section{TDP Annex}
The TDP's Robot Description Annex is an appendix of arbitrary length that must be attached at the end of the TDP and summarize the robot's software and hardware technical specifications.

The purpose of the annex is to provide an overview on how the robot operates.
This has several purposes.
First, it allows the league to track changes in hardware and software trends over time.
Second, it helps experienced teams to find alternatives to their solutions when looking to improve or conducting benchmarking.
Third, it serves as a quick reference guide for new teams while preparing their robots.
Finally, it helps the Technical Committee to keep the rulebook's specifications within the reach.

The annex must contain the following information:

\begin{itemize}[nosep]
	\item Photo(s) of the robot
	\item \textbf{OPL only:} Brief, compact description of the robot's hardware.
	\begin{itemize}[nosep]
		\item Robot's dimensions and weight.
		\item Traction type (base, e.g.~differential pair, omnidirectional, synchro-drive).
		\item Manipulators, count and number of DoF.
		\item Torsos (pan-elevation unit supporting the head), count and number of DoF.
		\item Heads (pan-tilt unit with a camera and optionally a microphone), count and number of DoF.
		\item Number and type of LIDAR sensors.
		\item Number and type of RGB-D sensors.
		\item Number and type of cameras.
		\item Number and type of microphones.
		\item Number and type of other sensors.
		\item Number and type of other actuators.
	\end{itemize}
	\item \textbf{DSPL \& SSPL:} Do \textbf{NOT} include hardware description.
	\item Brief, compact description of the robot's software (including commercial products, freeware, Open Source, etc.).
	\begin{itemize}[nosep]
		\item Audio filtering
		\item Automated speech recognition
		\item Manipulation
		\item Natural Language Processing
		\item Navigation, localization, and mapping
		\item Object recognition
		\item People recognition
		\item People tracking
		\item Pose/Gesture recognition
		\item Sound source localization
	\end{itemize}
	\item List of all external computing devices and the software running on them.
	\item List of all cloud computing resources intended to be used.
	\item Brief, compact description of all external devices (e.g.~smart home devices, transceivers, helper robots, etc.).
\end{itemize}

Examples are provided at the end of this document in
\hyperlink{page.9}{page~9~(DSPL)},
\hyperlink{page.10}{page~10~(OPL)}, and
\hyperlink{page.11}{page~11~(SSPL)}.


\section{Background}
% We are Buy n Large. We have no competitors so no background is required.
\lipsum[1-2]

\section{BnL Trash Seeker Algorithm (Main research)}
\lipsum[4-5]

\section{BnL All-purpose Speech Recognizer (Main research)}
\lipsum[9-10]

\section{Other relevant contributions}
\lipsum[12]
\subsection{Dirt Detector Algorithm}
\lipsum[13]
\subsection{Green Plant Seeker Algorithm}
\lipsum[14]
\subsection{Trash Seeker Algorithm}
\lipsum[15]

\section{Experiments and results}
\lipsum[23-24]

\section{Conclusions and future work}
\lipsum[25]

%%%%%%%%%%%%%%%%%%%%%%%%%%%%%%%%%%%%%%%%%%%%%%%%%%%%%%%%%%%%%%%%%%%%%%%%%%%%%%%%%%%%
%
% Bibliography
%
%%%%%%%%%%%%%%%%%%%%%%%%%%%%%%%%%%%%%%%%%%%%%%%%%%%%%%%%%%%%%%%%%%%%%%%%%%%%%%%%%%%%

\bibliographystyle{unsrt}
\bibliography{bibliography}

% Comment or delete the following line
{\vfill\begin{center}\huge\color{red}TDP must \textbf{NOT} exceed 8 pages\end{center}}

%%%%%%%%%%%%%%%%%%%%%%%%%%%%%%%%%%%%%%%%%%%%%%%%%%%%%%%%%%%%%%%%%%%%%%%%%%%%%%%%%%%%
%
% Robot Specifications
%
%%%%%%%%%%%%%%%%%%%%%%%%%%%%%%%%%%%%%%%%%%%%%%%%%%%%%%%%%%%%%%%%%%%%%%%%%%%%%%%%%%%%

\robospecs
\section*{EVA Software and External Devices [DSPL Template]}
% In this section briefly describe the software and hardware of the robot

\setlength\intextsep{0pt}
\begin{wrapfigure}[10]{r}{0.3\textwidth}
	\centering
	\includegraphics[width=0.4\textwidth]{images/eva.png}
	\caption{Robot EVA}
	\label{fig:eva}
\end{wrapfigure}

We use a standard EVA robot from \textit{Buy'N Large}. No modifications have been applied.

\section*{Robot's Software Description}
% Please describe in this section the software you are using to control your robot. Consider the following example:

\textit{For our robot we are using the following software:}

\begin{itemize}
	\item Platform: \BnL Operating System
	\item Face recognition: None. Not designed for human interaction.
	\item Object recognition: \BnL Green Plant Seeker Algorithm (See previous sections).
	\item Arms control and two-hand coordination: \BnL automatic controller \cite{bnl2}.
\end{itemize}

\section*{External Devices}
% Please describe in this section the external devices used by your robot. Consider the following example:

\textit{EVA robot relies on the following external hardware:}

\begin{itemize}
	\item \BnL Mother-ship
	\item \BnL Data Cluster
	\item $3 \times$ \BnL Ultra-Power laptops.
\end{itemize}

\section*{Cloud Services}
% Please describe in this section the Cloud Services and online software used by your robot. Consider the following example:

\textit{EVA connects the following cloud services:}
\begin{itemize}
	\item Localization and mapping: \BnL Geolocalization system \cite{bnl3}.
	\item Navigation: \BnL Navigator
	\item Speech recognition: \BnL All-purpose recognizer \cite{bnl1}.
	\item Speech generation: \BnL Speech synthesizer.
\end{itemize}

\newpage
\section*{Robot WALL-E Hardware Description [OPL Template]}
% In this section briefly describe the software and hardware of the robot

\setlength\intextsep{0pt}
\begin{wrapfigure}[10]{r}{0.3\textwidth}
	\centering
	\includegraphics[width=0.4\textwidth]{images/wall-e.jpg}
	\caption{Robot WALL-E}
	\label{fig:wall-e}
\end{wrapfigure}

Robot WALL-E has the patented \textit{\BnL Optimized Design} for garbage recollection. Specifications are as follows:

\begin{itemize}
	\item Base: \BnL all-terrain base (differential pair), 2.5m/s max speed.
	\item Torso: \BnL compressor with solar charger.
	\item Left and right arms: Mounted on torso. \BnL 7DOF, anthropomorphic. Maximum load: 20kg.
	\item Neck: \BnL telescopic neck with pan and tilt.
	\item Head: 3DOF \BnL Expressive Eyes
	\item Robot dimensions: height: 1.2m (max), width: 0.7m depth 0.8m
	\item Robot weight: 50kg.
\end{itemize}

\noindent\textit{Also our robot incorporates the following devices:}

\begin{itemize}
	\item \BnL Battery charge indicator
	\item \BnL Auto-focus all-purpose cameras
	\item \BnL 7DOF heavy duty fingers
	\item \BnL Cockroach
\end{itemize}

\section*{Robot's Software Description}
% Please describe in this section the software you are using to control your robot. Consider the following example:

\textit{For our robot we are using the following software:}

\begin{itemize}
	\item Platform: \BnL Operating System
	\item Navigation: \BnL Navigator
	\item Face recognition: None. Not designed for human interaction.
	\item Speech recognition: \BnL All-purpose recognizer \cite{bnl1}.
	\item Speech generation: None. Not designed for human interaction.
	\item Object recognition: \BnL Trash Seeker Algorithm (See previous sections).
	\item Arms control and two-hand coordination: \BnL automatic controller \cite{bnl2}.
\end{itemize}

\section*{External Devices}
% Please describe in this section the external devices used by your robot. Consider the following example:

\textit{WALL-E robot relies on the following external hardware:}

\begin{itemize}
	\item \BnL Garbage Compactor
	\item \BnL EVA unit
	\item \BnL Data Cluster
	\item $3 \times$ \BnL Ultra-Power laptops.
\end{itemize}

\section*{Cloud Services}
% Please describe in this section the Cloud Services and online software used by your robot. Consider the following example:

\textit{WALL-E connects the following cloud services:}
\begin{itemize}
	\item Localization and mapping: \BnL Geolocalization system \cite{bnl3}.
\end{itemize}
\newpage
\section*{M-O Software and External Devices [SSPL Template]}
\label{sec:annex-SSPL}
% In this section briefly describe the software and hardware of the robot

\setlength\intextsep{0pt}
\begin{wrapfigure}[10]{r}{0.3\textwidth}
	\centering
	\includegraphics[width=0.4\textwidth]{images/m-o.png}
	\caption{Robot M-O}
	\label{fig:m-o}
\end{wrapfigure}

We use a standard \textit{Buy'N Large} M-O robot unit. To differntiate our unit, an orange marker has been added on its top.

\section*{Robot's Software Description}
% Please describe in this section the software you are using to control your robot. Consider the following example:

\textit{For our robot we are using the following software:}

\begin{itemize}
	\item Platform: \BnL Operating System
	\item Face recognition: None. Not designed for human interaction.
	\item Speech generation: None. Not designed for human interaction.
	\item Object recognition: \BnL Dirt Detector Algorithm (See previous sections).
	\item Mop unit: \BnL automatic controller \cite{bnl2}.
\end{itemize}

\section*{External Devices}
% Please describe in this section the external devices used by your robot. Consider the following example:

\textit{M-O robot relies on the following external hardware:}

\begin{itemize}
	\item \BnL Mother-ship
	\item \BnL Data Cluster
	\item $3 \times$ \BnL Ultra-Power laptops.
\end{itemize}

\section*{Cloud Services}
% Please describe in this section the Cloud Services and online software used by your robot. Consider the following example:

\textit{M-O connects the following cloud services:}
\begin{itemize}
	\item Localization and mapping: \BnL Geolocalization system \cite{bnl3}.
	\item Navigation: \BnL Navigator
	\item Speech recognition: \BnL All-purpose recognizer \cite{bnl1}.
\end{itemize}

\nocite{*}

\end{document}
